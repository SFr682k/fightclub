\documentclass[11pt]{ltxdoc}

\usepackage{fontspec}
\usepackage{polyglossia}
\setdefaultlanguage{english}
\usepackage[english]{selnolig}

\setmainfont{Free Serif}
\setsansfont[Scale=MatchLowercase]{Free Sans}
\setmonofont[Scale=MatchLowercase]{Hack}

\usepackage[DIV=9]{typearea}

\usepackage{csquotes}
\usepackage{enumitem}
    \setlist{itemsep=0ex,topsep=.75ex}

\usepackage[table]{xcolor}


\usepackage{imakeidx}
\usepackage[unicode, pdfborder={0 0 0}, hyperindex=true, linktoc=all, pdfpagelabels]{hyperref}


\def\highlight#1{%
    \colorbox{red!12}{#1}%
    \index{\textsf{#1}}}


\MakeShortVerb{"}
\makeindex

\parindent 0pt

\hypersetup{pdftitle={Fightclub Client: Requirements on files}, pdfauthor={Sebastian Friedl}}


\title{\bfseries Fightclub Client: Requirements on files}
\author{Sebastian Friedl}
\date{\itshape Written to match the requirements of version 0.3}

\begin{document}
    \maketitle
    
    \begin{abstract}\noindent
        Fightclub Client uses plain text files for loading data. They can be created and edited using a common text editor; this file describes the syntactical requirements on such data files.
    \end{abstract}

    \tableofcontents
    
    
    
    
    
    
    
    \clearpage
    \section{General notes}
    Fightclub Client uses plain text files for loading data. They can be created and edited using a common text editor; however, those files require use of the respective file extension listed below and have to be encoded using "UTF-8" (without Byte Order Mark).

    \medskip
    To avoid loading wrong files -- and the resulting malfunction -- a \highlight{file header}, consisting of some "XML"-styled code, may be added to the first lines of a data file:
    
    \smallskip
    "<FightclubExchangeFile>stages</FightclubExchangeFile>" \\
    "<ExchangeProtocolVersion>1</ExchangeProtocolVersion>" \\
    "<ExchangeFileTitle>Stages for some tournament</ExchangeFileTitle>" \\
    "<ExchangeFileContentDescr>A short description</ExchangeFileContentDescr>"
    
    \bigskip
    \textbf{Note:}
    \begin{itemize}
        \item
            The tags are \textit{case sensitive}: "<FightclubExchangeFile>" gets recognized, while other variations like "<FIGHTCLUBexchangefile>" do not.
        \item
            Line breaks inside a tag are \textit{not} allowed; the whole tag must be opened and closed on the same line.
        
        \item
            The values of "<FightclubExchangeFile>" and "<ExchangeProtocolVersion>" depend on the exchange file and are listed below.
    \end{itemize} 
    
    \medskip
    As long as not being stated otherwise, \highlight{\texttt{\textbackslash t}} represents a tabstopp. It is \textit{not} possible to use spaces for indentation.
    
    
    
    
    
    \section{Requirements on data files}
    This section depicts the format of the exchange files. \\
    Examples are stored in the "sample-files" directory.
    
    
    \subsection{Files specifying stages}
    \begin{center}
        \rowcolors{1}{black!20}{white}
        \begin{tabular}{ll}
            File extension & ".fcstages" \\
            "<FightclubExchangeFile>" & "stages" \\
            "<ExchangeProtocolVersion>" & "1"
        \end{tabular}
    \end{center}
    
    
    There are two kinds of stages: stages of a physics fight and so-called \highlight{roomclock stages}, like an address of welcome or a lunch break. \\
    For each \textit{stage of a physics fight}, the list of phases will be gone through in the specified order while \textit{roomclock stages} only display their label without going through any phases.

    
    \subsubsection*{Syntax for physics fight stages}
    The syntax of a physics fight stage depends on the roles involved in it. It is
    \begin{center}
        "stage label  \t  problem  \t  rep. id  \t  opp. id  \t  rev. id"
    \end{center}
    for stages featuring all roles (Reporter, Opponent, Reviewer),
    \begin{center}
        "stage label  \t  problem  \t  rep. id  \t  opp. id"
    \end{center}
    for stages without reviewer and
    \begin{center}
        "stage label  \t  problem  \t  rep. id"
    \end{center}
    for stages with only a reporter.
    
    \bigskip
    \textit{Key:}
    \begin{itemize}
        \item
            "stage label": should be used to indicate fight, room and stage nr (e.g. "PF 3/D1" for stage 1 of physics fight 3 in room D or just a simple time like "09:00")
        \item
            "problem": either the problem's number or "-1" (when specifying "-1", the problem may be selected from a list)
 
        \item
            "rep. id", "opp. id", "rev. id": either the \textit{team id} of the reporting, opposing or reviewing team or the \textit{personal id} of the Reporter, Opponent or Reviewer
    \end{itemize}
    
    \medskip
    Please mind the difference between a \textit{stage} and a \textit{physics fight:} \\
    If there are three teams participating in a physics fight, there have to be three stages -- and therefore three lines in the stages file.
    
    \bigskip
    Teams with the Observer's role are not listed in such a stages file. \\
    Example for a four-teams-fight between "hun", "pol", "chn" and "sgp" with an observer: \\[\smallskipamount]
    "Final/Stage 1  \t  9   \t  hun  \t  pol  \t  chn" \\
    "Final/Stage 2  \t  17  \t  pol  \t  chn  \t  sgp" \\
    "Final/Stage 3  \t  3   \t  chn  \t  sgp  \t  hun" \\
    "Final/Stage 4  \t  5   \t  sgp  \t  hun  \t  pol" 
    
    
    
    \subsubsection*{Syntax for roomclock stages}
    For specifying a roomclock stage, simply add a single-column line to the stages file.
    
    
    \subsubsection*{Example lines for a \texttt{stages} file}
    "Address of Welcome" \\
    "PF 3/D1  \t  -1  \t  ger     \t  cze   \t  uk" \\
    "D 15:40  \t  17  \t  aggr/2  \t  dch3" \\
    "14:50    \t  9   \t  wshp/4" 
    
    
    
    \subsection{Files specifying phases}
    \begin{center}
        \rowcolors{1}{black!20}{white}
        \begin{tabular}{ll}
            File extension & ".fcphases" \\
            "<FightclubExchangeFile>" & "phases" \\
            "<ExchangeProtocolVersion>" & "1"
        \end{tabular}
    \end{center}
    
    
    \subsubsection*{General syntax}
    \begin{center}
        "duration  \t  overtime  \t  title  \t  performances  \t  options"
    \end{center}
    
    \smallskip
    \textit{Key:}
    \begin{itemize}
        \item "duration": the maximum duration of the phase in seconds
        \item "overtime": the maximum allowed overtime in seconds
        \item "title": the title of the phase as displayed
        \item "performances": the roles performing during this phase
        \item "options": further configuration of the phase
    \end{itemize}

    \smallskip
    You may omit
    \begin{itemize}
        \item
            the "options" column or
        \item
            the "performances" column together with the "options" column, but \textit{not} the "performances" column alone.
    \end{itemize}
    
    \smallskip
    Lines consisting of less than three tabular-separated columns are treated as comments and ignored.

    
    \subsubsection*{Allowed values for \texttt{performances}}
    \begin{itemize}
        \item "rep": the Reporter is performing during the current phase
        \item "opp": the Opponent is performing during the current phase
        \item "rev": the Reviewer is performing during the curr
        \item "nll": none of the three listed above are performing in the current phase
    \end{itemize}

    You have to specify \textit{at least one} of these values, including "nll". Values can be combined, e.g. one may use "repopp" for the discussion phase.

    \subsubsection*{Allowed \texttt{options}}
    \begin{itemize}
        \item "a": autoadvance to the next phase if maximum duration \textit{and} allowed overtime elapsed
        \item "c": carry the \textit{whole elapsed} time to the next phase
        \item "o": only carry the \textit{elapsed overtime} to the next phase
        \item "r": show the roomclock. Use this option only for phases with no maximum duration.
        \item "n": do not use any \enquote{special} options and behave like a \enquote{normal} phase.
    \end{itemize}

    Again, you have to specify \textit{at least one} of these values, including "n".
    
    
    \subsubsection*{Example lines for a \texttt{phases} file}
    "720  \t 0  \t Presentation of the report                \t rep     \t n" \\
    "240  \t 0  \t The Opponent takes the floor              \t opp     \t c" \\
    "840  \t 0  \t Discussion between Reporter and Opponent  \t repopp  \t n" \\
    "120  \t 0  \t Preparation of the Reviewer               \t rev     \t n" \\
    "600  \t 0  \t Break                                     \t nll     \t r"
    
    
    
    \subsection{Files specifying problems}
    \begin{center}
        \rowcolors{1}{black!20}{white}
        \begin{tabular}{ll}
            File extension & ".fcproblems" \\
            "<FightclubExchangeFile>" & "problems" \\
            "<ExchangeProtocolVersion>" & "1"
        \end{tabular}
    \end{center}
    
    
    \subsubsection*{General syntax}
    \begin{center}
        "nr  \t  title"
    \end{center}

    The list should be sorted by problem numbers in ascending order.
    
    \bigskip
    \textit{Key:}
    \begin{itemize}
        \item "nr": the number of the problem; requires an integer value larger than 0
        \item "title": the title of the problem
    \end{itemize}
    
    \smallskip
    Lines consisting of less than three tabular-separated columns are treated as comments and ignored.
    
    
    \subsubsection*{Example lines for a \texttt{problems} file}
    "5   \t  Leidenfrost Stars" \\
    "10  \t  Pulling Glasses Apart" \\
    "12  \t  Torsion Gyroscope"
    
    \medskip
    \textit{This is only an extract; a full \texttt{problems} file should of course contain all 17 problems ;)}
    
    
    
    
    

    \clearpage
    \printindex
    \phantomsection
    \addcontentsline{toc}{section}{Index}
\end{document}


